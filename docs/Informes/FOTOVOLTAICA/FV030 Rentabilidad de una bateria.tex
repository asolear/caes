%%%%%%%%%%%%%%%%%%%%%%%%%%%%%%%%%%%%%%%%%%%%%%%%%%%%%%%%%%%%%%%%%%%%%%%%%%%%%%%
\documentclass{article}

\newcommand{\path}{../../assets/settings}
\input{\path/newcommand.tex}
\input{\path/usepackage.tex}

\input{\path/quitarnumerossecciones.tex}


% \usepackage{nopageno} % Paquete para desactivar la numeración de páginas

\title{Resultados Instalacion FV}
\author{Kgnete}
\date{\today}

\begin{document}

\maketitle
%%%%%%%%%%%%%%%%%%%%%%%%%%%%%%%%%%%%%%%%%%%%%%%%%%%%%%%%%%%%%%%%%%%%%%%%%%%%%%%

\section*{rentabilidad de una batería}



Para determinar la rentabilidad de una batería de 1 kWh con una vida útil de 6000 ciclos, seguimos el mismo proceso de cálculo:

Costo de carga por ciclo:

Capacidad de la batería: 1 kWh

Costo de carga por kWh: 0.07 euros/kWh

Costo de carga por ciclo
=
1
x
kWh
×
0.07
x
euros/kWh
=
0.07
x
euros

Costo de carga por ciclo=1kWh×0.07euros/kWh=0.07euros
Ingreso de descarga por ciclo:


Capacidad de la batería: 1 kWh

Ingreso de descarga por kWh: 0.15 euros/kWh

Ingreso de descarga por ciclo
=
1
x
kWh
×
0.15
x
euros/kWh
=
0.15
x
euros

Ingreso de descarga por ciclo=1kWh×0.15euros/kWh=0.15euros

Beneficio neto por ciclo:

Ingreso de descarga por ciclo: 0.15 euros

Costo de carga por ciclo: 0.07 euros

Beneficio neto por ciclo
=
0.15
x
euros

0.07
x
euros
=
0.08
x
euros

Beneficio neto por ciclo=0.15euros-0.07euros=0.08euros

Beneficio total durante la vida útil de la batería:

Número de ciclos: 6000 ciclos
Beneficio neto por ciclo: 0.08 euros
Beneficio total
=
6000
x
ciclos
×
0.08
x
euros/ciclo
=
480
x
euros



Beneficio total=6000ciclos×0.08euros/ciclo=480euros

$2 \times 3$ = 

\ADD{3}{5}{\suma}

\suma

\subsubsection*{Resumen}

%%

La rentabilidad total de una batería de 1 kWh durante su vida útil de 6000 ciclos es de 480 euros. Este valor representa el beneficio neto que se obtiene al cargar la batería a 0.07 euros/kWh y descargarla a 0.15 euros/kWh.




\begin{thebibliography}{9}
    \bibitem{1}
    IDAE. 
    Instalaciones de
Energía Solar Fotovoltaica.
Pliego de Condiciones Técnicas de
Instalaciones Conectadas a Red
PCT-C-REV - julio 2011

\end{thebibliography}

\end{document}


