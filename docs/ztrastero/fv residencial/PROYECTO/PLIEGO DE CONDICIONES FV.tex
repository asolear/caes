\documentclass{article}

\input{../../../assets/settings/newcommand.tex}
\input{../../../assets/settings/usepackage.tex}
% \input{../../../assets/settings/quitarnumerossecciones.tex}

% \usepackage{nopageno} % Paquete para desactivar la numeración de páginas

\title{PLIEGO DE CONDICIONES}
\author{KGNETE}
\date{\today}

\begin{document}

\maketitle
\section{PLIEGO DE CONDICIONES}

\subsection{CARECTERÍSTICAS DE LA EMPRESA INSTALADORA}
Las instalaciones eléctricas de baja tensión serán ejecutadas por la empresa instaladora
autorizada, contando para ello con instalador Autorizado en Baja Tensión, autorizado para el
ejercicio de la actividad según lo establecido en la correspondiente Instrucción Técnica
Complementaria del R.E.B.T., sin perjuicio de su posible proyecto y dirección de obra por
técnicos titulados pertenecientes a dicha empresa instaladora.




\subsection{CALIDAD DE LOS MATERIALES}
Todos los materiales a emplear en la presente instalación serán de primera calidad y reunirán
las condiciones exigidas en el Reglamento Electrotécnico para Baja Tensión y demás
disposiciones vigentes referentes a materiales y prototipos de construcción.
Todos los trabajos incluidos en el presente proyecto se ejecutarán con arreglo a las buenas
prácticas de las instalaciones eléctricas, de acuerdo con el Reglamento Electrotécnico para
Baja Tensión, y cumpliendo estrictamente las instrucciones recibidas por la Dirección
Facultativa, no pudiendo, por tanto, servir de pretexto al contratista la baja en subasta, para
variar esa esmerada ejecución ni la primerísimo calidad de las instalaciones proyectadas en
cuanto a sus materiales y mano de obra, ni pretender proyectos adicionales.
Es por ello que los elementos que se describen como posibles materiales a utilizar cumplen los
mínimos exigidos por los indicados en el proyecto. Si por motivos se utilizasen equipos
diferentes en la instalación, estos tendrían que ser de características equivalentes por las
indicadas en las fichas adjuntas en el anexo de equipos y calidades iguales o superiores.




\subsubsection{CONDUCTORES ELÉCTRICOS}
Los conductores utilizados se regirán por las especificaciones del proyecto, según se indica en
Memoria, Planos y Presupuesto.

El tipo de cable que se empleará será RV-K 0,6/1 kV, cuyas características técnicas son las que
se muestran a continuación:

Flama: No propagador de llama, UNE-20432.1 (IEC-332.1)

Conductor de Cu: Clase 5

Aislamiento: XLPE

Cubierta:
PVC

Temperatura máxima de utilización: 90 °C

Características constructivas: UNE-21 123 (P-2)

Los conductores de sección igual o superior a 6 mm² deberán estar constituidos por cable
obtenido por trenzado de hilo de cobre del diámetro correspondiente a la sección del conductor
de que se trate.

Para la selección de la sección de los conductores activos del cable adecuado a cada carga se
usará el más desfavorable entre los siguientes criterios:

- Intensidad máxima admisible. Como intensidad se tomará la propia de cada generador
fotovoltaico, partiendo de las intensidades nominales así establecidas, se elegirá la sección del
cable que admita esa intensidad de acuerdo a las prescripciones del Reglamento
Electrotécnico para Baja Tensión ITC-BT-19 o las recomendaciones del fabricante, adoptando
los oportunos coeficientes correctores según las condiciones de la instalación.

- Caída de tensión en servicio. La sección de los conductores a utilizar se determinará de forma
que la caída de tensión para la parte de continua no podrá ser superior al 1.5% y para la parte de
alterna no podrá ser superior al 1.5%.

La sección del conductor neutro será la especificada en la Instrucción ITC-BT-07, apartado 1,
en función de la sección de los conductores de fase o polares de la instalación.




\subsubsection{CONDUCTORES DE PROTECCIÓN}
Los conductores de protección serán del mismo tipo que los conductores activos
especificados en el apartado anterior, y tendrán una sección mínima a la fijada en la tabla 2 de
la ITC-BT-18, en función de la sección de los conductores de fase o polares de la instalación. Se
podrán instalar por las mismas canalizaciones que estos o bien en forma independiente.


\subsubsection{IDENTIFICACIÓN DE LOS CONDUCTORES}
Para la identificación de los conductores en la parte de corriente continua se marcarán de
forma permanente el positivo de color Rojo y el negativo de color Azul, los colores de los
recubrimientos serán Azul para el neutro, Marrón, Gris o Negro para las fases y Amarillo-Verde
para los de protección.

Las canalizaciones eléctricas se establecerán de forma que por conveniente identificación de
sus circuitos y elementos, se pueda proceder en todo momento a reparaciones,
transformaciones, etc.



\subsubsection{CANALIZACIONES}
Las características de protección de la unión entre el tubo y sus accesorios no deben ser
inferiores a los declarados para el sistema de tubos.

La superficie interior de los tubos no deberá presentar en ningún punto aristas, asperezas o
fisuras susceptibles de dañar los conductores o cables aislados o de causar heridas a
instaladores o usuarios.

Las dimensiones de los tubos no enterrados y con unión roscada utilizados en las instalaciones
eléctricas son las que se prescriben en la UNE-EN 60.423.

Para los tubos enterrados, las dimensiones se corresponden con las indicadas en la norma UNE-
EN 50.086-2-4. Para el resto de los tubos, las dimensiones serán las establecidas en la norma
correspondiente de las citadas anteriormente. La denominación se realizará en función del
diámetro exterior. El diámetro interior mínimo deberá ser declarado por el fabricante.


En lo relativo a la resistencia a los efectos del fuego considerados en la norma particular para
cada tipo de tubo, se seguirá lo establecido por la aplicación de la Directiva de Productos de la
Construcción (89/106/CEE).

En las canalizaciones superficiales, los tubos deberán ser preferentemente rígidos y en casos
especiales podrán usarse tubos curvables. Sus características mínimas serán las indicadas en
ITC-BT-21.

En las canalizaciones empotradas, los tubos protectores podrán ser rígidos, curvables o
flexibles, con unas características mínimas indicadas en ITC-BT-21.

Los tubos en canalizaciones enterradas presentarán las características señaladas en ITC-BT-
21. El diámetro exterior mínimo de los tubos, en función del número y la sección de los
conductores a conducir, se obtendrá de las tablas indicadas en la ITC-BT-21, así como las
características mínimas según el tipo de instalación.

En general, para la ejecución de las canalizaciones bajo tubos protectores, se tendrá en cuenta
lo dictado en ITC-BT-21.

La canal protectora es un material de instalación constituido por un perfil de paredes
perforadas o no, destinado a alojar conductores o cables y cerrado por una tapa desmontable.
Las canalizaciones para instalaciones superficiales tendrán unas características mínimas
señaladas en apartado 3 de ITC-BT-21.

En bandeja o soporte de bandejas, sólo se utilizarán conductores aislados con cubierta,
unipolares o multipolares según norma UNE 20.460-5-52.

El material usado para la fabricación será acero laminado de primera calidad, galvanizado por
inmersión.

La anchura de las canaletas será de 100 mm como mínimo, con incrementos de 100 en 100 mm.
La longitud de los tramos rectos será de dos metros. El fabricante indicará en su catálogo la
carga máxima admisible, en N/m, en función de la anchura y de la distancia entre soportes.
Todos los accesorios como codos, cambios de plano, reducciones, tes, uniones, soportes, etc.
Tendrán la misma calidad que la bandeja.

La bandeja y sus accesorios se sujetarán a techos y paramentos mediante herrajes de
suspensión, a distancias tales que no se produzcan flechas superiores a 10 mm. Y estarán
perfectamente alineadas con los cerramientos de los locales.

No se permitirá la unión entre bandejas o la fijación de las mismas a los soportes por medio de
soldadura, debiéndose utilizar piezas de unión y tornillería cadmiada. Para las uniones o
derivaciones de líneas se utilizarán cajas metálicas que se fijarán a las bandejas.


\subsubsection{CAJAS DE EMPALME Y DERIVACIÓN}
Las conexiones entre conductores se realizarán en el interior de cajas apropiadas de material
plástico resistente incombustible o metálicas, en cuyo caso estarán aisladas interiormente y
protegidas contra la oxidación. Las dimensiones de estas cajas serán tales que permitan alojar
holgadamente todos los conductores que deban contener. Su profundidad será igual, por lo
menos, a una vez y media el diámetro del tubo mayor, con un mínimo de 40 mm; el lado o
diámetro de la caja será de al menos 80 mm. Cuando se quieran hacer estancas las entradas de
los tubos en las cajas de conexión, deberán emplearse prensaestopas adecuados. En ningún
caso se permitirá la unión de conductores, como empalmes o derivaciones por simple
retorcimiento o arrollamiento entre sí de los conductores, sino que deberá realizarse siempre
utilizando bornes de conexión.

Los conductos se fijarán firmemente a todas las cajas de salida, de empalme y de paso,
mediante contratuercas y casquillos. Se tendrá cuidado de que quede al descubierto el número
total de hilos de rosca al objeto de que el casquillo pueda ser perfectamente apretado contra el
extremo del conducto, después de lo cual se apretará la contratuerca para poner firmemente el
casquillo en contacto eléctrico con la caja.

Los conductos y cajas se sujetarán por medio de pernos de fiador en ladrillo hueco, por medio
de pernos de expansión en hormigón y ladrillo macizo y clavos Split sobre metal. Los pernos de
fiador de tipo tornillo se usarán en instalaciones permanentes, los de tipo de tuerca cuando se
precise desmontar la instalación, y los pernos de expansión serán de apertura efectiva. Serán
de construcción sólida y capaz de resistir una tracción mínima de 20 kg. No se hará uso de
clavos por medio de sujeción de cajas o conductos.



\subsubsection{APARATOS DE MANDO Y MANIOBRA}
Las únicas maniobras posibles en las centrales solares fotovoltaicas son las de puesta en
marcha y parada de los Inversores que forman el generador fotovoltaico.

Para gobierno y maniobra del inversor instalado, se dispondrán además de los correspondientes
elementos de protección, elementos de seccionamiento en la parte de corriente continua y un
interruptor de corte en la parte de corriente alterna que garanticen la ausencia de tensión en
bornes de cada inversor.


\subsubsection{APARATOS DE PROTECCIÓN}
\end{document}
