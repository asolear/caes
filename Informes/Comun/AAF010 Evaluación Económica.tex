\title{Informe Técnico \\ Evaluación Económica de la Inversión  }

\newcommand{\resumen}{    Este informe presenta una evaluación económica de la inversión mediante el análisis de tres indicadores clave: 
Tasa de Periodo de Retorno (PR), Valor Actual Neto (VAN) y Tasa Interna de Rentabilidad (TIR). 
El objetivo es determinar la viabilidad y rentabilidad del proyecto en función de los flujos de caja esperados a lo 
largo del tiempo, así como las tasas de retorno asociadas.}

% Declarar los datos como una tabla simple
% % %%%%%%%%%%%%%%%%%%%%%%%%%%%%%%%%%%%%%%%%%%%%%%%
% % %%%%%%%%%%%%%%%%%%%%%%%%%%%%%%%%%%%%%%%%%%%%%%%
% % %%%%%%%%%%%%%%%%%%%%%%%%%%%%%%%%%%%%%%%%%%%%%%%
% % %%%%%%%%%%%%%%%%%%%%%%%%%%%%%%%%%%%%%%%%%%%%%%%
% % %%%%%%%%%%%%%%%%%%%%%%%%%%%%%%%%%%%%%%%%%%%%%%%
\documentclass[conference,12pt]{IEEEtran}
\usepackage[pdftex,pdfencoding=auto]{hyperref}
\usepackage[utf8]{inputenc} % Allows UTF-8 input
\usepackage{amsmath} % For mathematical equations
\usepackage{lipsum} % Para texto falso
\usepackage{tikz} % Para diagramas
\usepackage{float}
\usepackage{ragged2e} % Para usar la justificación del texto
\usepackage[T1]{fontenc}    % Para una codificación de fuente adecuada
\usepackage{colortbl} % Paquete para colores en tablas
\usepackage{pgfplots}
\usepackage[spanish]{babel} % produce error con las flechas de los tikz
\usepackage{caption}
\usepackage{pgffor}  % Paquete para bucles
\usepackage{fancyhdr} % Para personalizar encabezados y pies de página
% %%%%%%%%%%%%%%%%%%%%%%%%%%%%%%%%%%%%%%%%%%%%%%%
\setlength{\spaceskip}{4.8pt}
% Redefinir la numeración de secciones, subsecciones y subsubsecciones
\renewcommand{\thesection}{\arabic{section}} % 1, 2, 3...
\renewcommand{\thesubsection}{\thesection.\arabic{subsection}} % 1.1, 1.2...
\renewcommand{\thesubsubsection}{\thesubsection.\arabic{subsubsection}} % 1.1.1, 1.1.2...
\let\OldTextField\TextField
\renewcommand{\TextField}[2][]{%
  \raisebox{-0.3ex}{\OldTextField[height=.85em,  bordercolor={1 1 1}, backgroundcolor={1 1 1},#1]{#2}}%
}
\renewcommand{\baselinestretch}{1.5}  % 1.5 es el valor estándar, pero puedes aumentarlo a 2, 2.5, etc.
\renewcommand{\rmdefault}{phv}  % Cambia la fuente de texto a Helvetica
\fontsize{12}{15}\selectfont  % Establece el tamaño de la fuente y la altura de línea
\onecolumn
% Configuración del pie de página
\pagestyle{fancy}
\fancyhf{} % Limpia cabeceras y pies de página
\fancyfoot[C]{\thepage} % Centra el número de página en el pie
% Eliminar líneas en cabecera y pie
\renewcommand{\headrulewidth}{0pt} % Sin línea en la cabecera
\renewcommand{\footrulewidth}{0pt} % Sin línea en el pie
%%%%%%%%%%%%%%%%%%%%%%%%%%%%%%%%%%%%%%%%%%%%%%%
\author{\TextField[name=Proyecto,width=16cm]{}}
\date{\today}

%%%%%%%%%%%%%%%%%%%%%%%%%%%%%%%%%%%%%%%%%%%%%%%
\begin{document}
%%%%%%%%%%%%%%%%%%%%%%%%%%%%%%%%%%%%%%%%%%%%%%%

\justifying
\begin{Form}
\maketitle
\vspace{10cm}
%%%%%%%%%%%%%%%%%%%%%%%%%%%%%%%%%%%%%%%%%%%%%%%
\begin{table}[h!]
    \centering
    \begin{tabular}{|p{2cm}|p{8cm}|}
        \hline
        Técnico: &     \TextField[name=Tecnico,width=6cm]{} 
        \\
        \hline
        Organización: &     \TextField[name=Organizacion,width=6cm]{} 
        \\
        \hline
        NIF/NIE: &    \TextField[name=NIF,width=6cm]{} 
        \\ 
        \hline
        Fecha: &    \TextField[name=Fecha,width=6cm]{} 
        \\ 
        \hline
        Firma: &     
        \\ &
        \\ &
        \\ &
        \\ \hline
    \end{tabular}
\end{table}
%%%%%%%%%%%%%%%%%%%%%%%%%%%%%%%%%%%%%%%%%%%%%%%
\newpage
\begin{abstract}
    \resumen
\end{abstract}
\tableofcontents
\newpage
% \newpage
% % %%%%%%%%%%%%%%%%%%%%%%%%%%%%%%%%%%%%%%%%%%%%%%%
% % %%%%%%%%%%%%%%%%%%%%%%%%%%%%%%%%%%%%%%%%%%%%%%%
% % %%%%%%%%%%%%%%%%%%%%%%%%%%%%%%%%%%%%%%%%%%%%%%%
% % %%%%%%%%%%%%%%%%%%%%%%%%%%%%%%%%%%%%%%%%%%%%%%%
% % %%%%%%%%%%%%%%%%%%%%%%%%%%%%%%%%%%%%%%%%%%%%%%%


\section{Introducción}

La evaluación económica considera inversión inicial del proyecto y los beneficios económicos derivados del mismo. A continuación, 
se detalla algunas reglas generalmente utilizadas para decidir.

\section{Indicadores Financieros}

\subsection{Periodo de Retorno (PR)}

El PR es el tiempo  necesario para recuperar la inversión inicial:

$$
PR=\frac{I_0}{B_t}
$$

\subsection{Valor Actual Neto (VAN)}

El VAN calcula el valor presente de los flujos de caja futuros descontados, menos la inversión inicial. Representa el valor que la inversión agrega en términos monetarios.

$$
VAN = \sum_{t=0}^{n} \frac{B_t}{(1+i)^n} - I_0
$$




% \begin{figure}[H]
%     \centering
% \begin{tikzpicture}[scale=.5, every node/.style={font=\small}]
%     % Ejes
%     \draw[->] (0,0) -- (10,0) node[right]{Tasa de descuento (\%)};
%     \draw[->] (0,-2) -- (0,4) node[above]{VAN};


%     \draw[thick, blue] (0,3) to[out=0, in=180] (9,-2);

% \end{tikzpicture}
% \caption{Curva de TIR: Punto de intersección donde VAN = 0.}
% \label{fig:tir}
% \end{figure}

\subsection{Tasa Interna de Retorno (TIR)}

La Tasa Interna de Retorno (TIR) es la tasa de descuento que hace que el VAN sea igual a cero. Se calcula resolviendo la siguiente ecuación:

$$
    \sum_{t=0}^{n} \frac{B_t}{(1+TIR)^t} = I_0
$$




\begin{figure}[H]
        \centering
    \begin{tikzpicture}[scale=.5, every node/.style={font=\small}]
        % Ejes
        \draw[] (0,0) -- (10,0) node[right, align=center] {i};
        \draw[] (0,-2) -- (0,4) node[above]{VAN};


        % Spline usando to[out=angle1, in=angle2]
        \draw[thick, blue] (.5,4) to[out=-50, in=150] (4,0);
        \draw[thick, blue] (4,0) to[out=-30, in=170] (9,-1);

        % Sombrear el área entre la curva y el eje X
        % \fill[blue, opacity=0.1] (4,0) to[out=-30, in=170] (9,-1) -- (9,0) -- (4,0) -- cycle;

   
        % Punto TIR
        \node[red] at (4,0) {\textbullet}; % Punto donde VAN = 0
        \node[below] at (4,0) {TIR};

        % Etiquetas adicionales
        % \node[above] at (2,2) {VAN positivo};
        % \node[below] at (7,-1.5) {VAN negativo};

        % Líneas guía opcionales
        % \draw[dashed] (5,0) -- (5,-1) node[below] {5\%};
    \end{tikzpicture}
    \caption{Curva de TIR: Punto de intersección donde VAN = 0.}
    \label{fig:tir}
\end{figure}


Donde:
\begin{itemize}
    \item $I_0$: Inversión inicial [EUR].
    \item $B_t$: Beneficio neto anual en el año.\footnote{Supuesto constante durante un número de periodos,.} [EUR/a].

    \item $i$: Tasa de descuento.[\%]
    \item $n$: Número de periodos del análisis [años].
    \item $PR$: Periodo de Retorno. [años].
    \item $VAN$: Valor Actual Neto.[EUR].
    \item $TIR$: Tasa Interna de Retorno.[\%]
\end{itemize}

\section{Resultados}


\begin{table}[H]
    \centering
    \begin{tabular}{
        |p{1.20cm}
        |p{1.2cm}
        |p{.5cm}
        |p{.5cm}
        |p{.7cm}
        |p{1.5cm}
        |p{.7cm}
        |}
    \hline
     $I_0$ &
     $B_t$ &
     i &
     n & 
     \cellcolor{gray!30}  PR &
     \cellcolor{gray!30}  VAN &
     \cellcolor{gray!30}  TIR \\ 
     \hline
     \TextField[name=I0,width=1.2cm]{} & 
     \TextField[name=Bt,width=1.2cm]{} &
     \TextField[name=i,width=.5cm]{} &
     \TextField[name=n,width=.5cm]{} &
     \TextField[name=PR,width=.7cm]{} &
     \TextField[name=VAN,width=1.5cm]{} &
     \TextField[name=TIR,width=.7cm]{} 
    \\ \hline
    \end{tabular}
\end{table}


\section{Conclusiones}

La inversion propuesta 
\TextField[name=evaluacion,width=3cm]{}


La siguiente tabla resume los criterios para valorar la conveniencia de una inversión:

\begin{table}[h!]
    \centering
    \begin{tabular}{|>{\raggedright\arraybackslash}p{2cm}|>{\raggedright\arraybackslash}p{2cm}|>{\raggedright\arraybackslash}p{4cm}|}
        \hline
        \textbf{Indicador} & \textbf{Conveniente si...} & \textbf{Uso principal} \\ \hline
        \textbf{Período de Retorno(PR)} & Menor que el período objetivo & Evaluar el riesgo y el tiempo necesario para recuperar la inversión. \\ \hline
        \textbf{Valor Actual Neto (VAN)} & Mayor que 0 & Determinar si la inversión genera valor económico adicional. \\ \hline
        \textbf{Tasa Interna de Retorno (TIR)} & Mayor que la tasa de descuento & Comparar la rentabilidad relativa con otras oportunidades o con el costo de capital. \\ \hline
    \end{tabular}
    \caption{Criterios de evaluación de una inversión.}
    \label{tabla:criterios}
\end{table}

















\section{Referencias}
    \begin{itemize}
        \item \href{https://sam.nrel.gov/financial-models.html}
        {SAM Financial Models}, Web: https://sam.nrel.gov/financial-models.html., NREL
    
    \end{itemize}
\end{Form}


\end{document}
