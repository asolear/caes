%%%%%%%%%%%%%%%%%%%%%%%%%%%%%%%%%%%%%%%%%%%%%%%%%%%%%%%%%%%%%%%%%%%%%%%%%%%%%%%
\documentclass{article}

\input{../../../assets/settings/newcommand.tex}
\input{../../../assets/settings/usepackage.tex}

\usepackage{nopageno} % Paquete para desactivar la numeración de páginas

\begin{document}

\section*{DETERMINACIÓN DE LA CAPACIDAD DISPONIBLE POR UN CONSUMIDOR
DOMÉSTICO PARA REALIZAR LA RECARGA DEL VE SIN AMPLIAR LA POTENCIA.}


El operador del sistema (Red Eléctrica de España), calcula y publica regularmente las medidas de la demanda del
sistema eléctrico peninsular y los perfiles finales de consumo. Gracias al proyecto perfila, estos perfiles de consumo
aplicables a los consumidores domésticos se han podido determinar con precisión.

En base a esta información, y con el objetivo de poder estimar de una manera razonable y robusta el margen de
capacidad libre o “hueco” que tendrían los consumidores domésticos para realizar la cargar nocturna del VE, se han
tomado los valores máximos para cada periodo horario del coeficiente de perfilado A publicado por REE durante el
año 2015. Estos valores, ajustados en base 100 para el valor máximo de dicho coeficiente horario, han sido
representados en la siguiente gráfica

\begin{figure}[H]
    \centering
    \includegraphics[width=0.8\linewidth]{diario_Consumo_kWh.png} % Ajusta la ruta y el tamaño según tus necesidades
    \caption{Emplazamiento geográfico.}
    \label{fig:etiqueta}
  \end{figure}

De esta manera, se obtiene el ratio horario de uso de la capacidad disponible por un consumidor doméstico.
Suponiendo que los VE fuera programados para que iniciaran su carga a partir de la 1 de la mañana (hora de inicio
de la tarifa de acceso supervalle, que coincide además con los precios más bajos de la energía en el mercado), un
consumidor doméstico tendría disponible en un escenario de máxima demanda para esta hora, prácticamente el
50\% de su capacidad de punta para poder realizar esta recarga.

En caso el de que se comprobara que los VE conectados a los puntos de recarga de las viviendas no realizan en su
mayoría una recarga lenta a partir de esta hora, este coeficiente debería ser recalculado.


\begin{thebibliography}{9}
    \bibitem{1}
    REBT. GUÍA ITC-BT 52

    \bibitem{2}
    
    \href{https://www.ree.es/es/clientes/generador/gestion-medidas-electricas/consulta-perfiles-de-consumo}{REE. Consulta los perfiles de consumo (TBD)}

    

\end{thebibliography}

\end{document}

