
% enlaces
% https://www.youtube.com/watch?v=IqSozXsITGc
% normas une
% https://plataforma-e.aenormas.aenor.com/
% https://plataforma-e.aenormas.aenor.com/?hash=$2b$10$a2v23HMhQ6DtxKZrtl7uce6ZIcvdKkEgsTtYPLgNEoARhNn0gMj3i

% \href{https://plataforma-e.aenormas.aenor.com/pdf/UNE/N0070393}{
%     norma-une-en-iso-52120-1-2022    


\documentclass[conference]{IEEEtran}
\usepackage[utf8]{inputenc} % Allows UTF-8 input
\usepackage{graphicx} % For images
\usepackage{amsmath} % For mathematical equations
\usepackage{lipsum} % Para texto falso
\usepackage[pdfencoding=auto]{hyperref} 
\let\OldTextField\TextField
\renewcommand{\TextField}[2][]{%
  \OldTextField[height=1.1em, bordercolor={1 1 1}, borderwidth=0, backgroundcolor={1 1 1 0},#1]{#2}%
}



\title{An Example IEEE Conference Paper}
\author{
    \IEEEauthorblockN{John Doe}
    \IEEEauthorblockA{
        Department of Computer Science\\
        University Name, City, Country \\
        Email: john.doe@example.com
    }
    \and
    \IEEEauthorblockN{Jane Smith}
    \IEEEauthorblockA{
        Department of Electrical Engineering\\
        Institute Name, City, Country \\
        Email: jane.smith@example.com
    }
}

\begin{document}
\begin{Form}

\maketitle

\begin{abstract}
This document demonstrates the IEEEtran class for typesetting conference papers. 
The abstract section summarizes the main points of the paper concisely.
\end{abstract}

\begin{IEEEkeywords}
IEEE, LaTeX, template, conference paper, example
\end{IEEEkeywords}

\section{Introduction}
This jojo  \TextField[name=S,width=1.1cm]{} is an example of a conference paper formatted using the \texttt{IEEEtran} class. The introduction section introduces the topic and sets the context for the paper.

\section{Methodology}
This section explains the methodology. Use equations like:
\begin{equation}
    E = mc^2
\end{equation}
to include mathematical expressions.

\subsection{Subsection Example}
Subsections can be used to organize content. This is an example of a subsection.

\section{Results}
Results are presented here. Use figures like:
\begin{figure}[ht]
    \centering
    \includegraphics[width=0.4\textwidth]{example-image}
    \caption{An example figure.}
    \label{fig:example}
\end{figure}


\section{Resultados}
Aquí se presentan resultados:
\begin{figure}[ht]
    \centering
    \includegraphics[width=0.4\textwidth]{example-image}
    \caption{Ejemplo de figura.}
    \label{fig:example}
\end{figure}
Más texto simulado:
\begin{equation}
    E = mc^2
\end{equation}
\lipsum[5]

\section{Conclusión}
Este documento utiliza el paquete \texttt{lipsum} para insertar texto simulado.
\lipsum[6]
\section{Conclusion}
The conclusion summarizes the key findings and provides insights into future work.

\section*{Acknowledgments}
The authors would like to thank XYZ for their support.

\bibliographystyle{IEEEtran}
\bibliography{references}

\end{Form}
\end{document}
