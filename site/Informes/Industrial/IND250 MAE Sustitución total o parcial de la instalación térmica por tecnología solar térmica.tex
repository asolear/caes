\title{Informe Técnico\\ \textbf{IND250 MAE Sustitución total o parcial de la instalación
térmica por tecnología solar térmica}}

\newcommand{\resumen}{    Este informe presenta una evaluación económica de la inversión mediante el análisis de tres indicadores clave: 
Tasa de Periodo de Retorno (PR), Valor Actual Neto (VAN) y Tasa Interna de Rentabilidad (TIR). 
El objetivo es determinar la viabilidad y rentabilidad del proyecto en función de los flujos de caja esperados a lo 
largo del tiempo, así como las tasas de retorno asociadas.}

% Declarar los datos como una tabla simple
% % %%%%%%%%%%%%%%%%%%%%%%%%%%%%%%%%%%%%%%%%%%%%%%%
% % %%%%%%%%%%%%%%%%%%%%%%%%%%%%%%%%%%%%%%%%%%%%%%%
% % %%%%%%%%%%%%%%%%%%%%%%%%%%%%%%%%%%%%%%%%%%%%%%%
% % %%%%%%%%%%%%%%%%%%%%%%%%%%%%%%%%%%%%%%%%%%%%%%%
% % %%%%%%%%%%%%%%%%%%%%%%%%%%%%%%%%%%%%%%%%%%%%%%%
\documentclass[conference,12pt]{IEEEtran}
\usepackage[pdftex,pdfencoding=auto]{hyperref}
\usepackage[utf8]{inputenc} % Allows UTF-8 input
\usepackage{amsmath} % For mathematical equations
\usepackage{lipsum} % Para texto falso
\usepackage{tikz} % Para diagramas
\usepackage{float}
\usepackage{ragged2e} % Para usar la justificación del texto
\usepackage[T1]{fontenc}    % Para una codificación de fuente adecuada
\usepackage{colortbl} % Paquete para colores en tablas
\usepackage{pgfplots}
\usepackage[spanish]{babel} % produce error con las flechas de los tikz
\usepackage{caption}
\usepackage{pgffor}  % Paquete para bucles
\usepackage{fancyhdr} % Para personalizar encabezados y pies de página
% %%%%%%%%%%%%%%%%%%%%%%%%%%%%%%%%%%%%%%%%%%%%%%%
\setlength{\spaceskip}{4.8pt}
% Redefinir la numeración de secciones, subsecciones y subsubsecciones
\renewcommand{\thesection}{\arabic{section}} % 1, 2, 3...
\renewcommand{\thesubsection}{\thesection.\arabic{subsection}} % 1.1, 1.2...
\renewcommand{\thesubsubsection}{\thesubsection.\arabic{subsubsection}} % 1.1.1, 1.1.2...
\let\OldTextField\TextField
\renewcommand{\TextField}[2][]{%
  \raisebox{-0.3ex}{\OldTextField[height=.85em,  bordercolor={1 1 1}, backgroundcolor={1 1 1},#1]{#2}}%
}
\renewcommand{\baselinestretch}{1.5}  % 1.5 es el valor estándar, pero puedes aumentarlo a 2, 2.5, etc.
\renewcommand{\rmdefault}{phv}  % Cambia la fuente de texto a Helvetica
\fontsize{12}{15}\selectfont  % Establece el tamaño de la fuente y la altura de línea
\onecolumn
% Configuración del pie de página
\pagestyle{fancy}
\fancyhf{} % Limpia cabeceras y pies de página
\fancyfoot[C]{\thepage} % Centra el número de página en el pie
% Eliminar líneas en cabecera y pie
\renewcommand{\headrulewidth}{0pt} % Sin línea en la cabecera
\renewcommand{\footrulewidth}{0pt} % Sin línea en el pie
%%%%%%%%%%%%%%%%%%%%%%%%%%%%%%%%%%%%%%%%%%%%%%%
\author{\TextField[name=Proyecto,width=16cm]{}}
\date{\today}

%%%%%%%%%%%%%%%%%%%%%%%%%%%%%%%%%%%%%%%%%%%%%%%
\begin{document}
%%%%%%%%%%%%%%%%%%%%%%%%%%%%%%%%%%%%%%%%%%%%%%%

\justifying
\begin{Form}
\maketitle
\vspace{10cm}
%%%%%%%%%%%%%%%%%%%%%%%%%%%%%%%%%%%%%%%%%%%%%%%
\begin{table}[h!]
    \centering
    \begin{tabular}{|p{2cm}|p{8cm}|}
        \hline
        Técnico: &     \TextField[name=Tecnico,width=6cm]{} 
        \\
        \hline
        Organización: &     \TextField[name=Organizacion,width=6cm]{} 
        \\
        \hline
        NIF\/NIE: &    \TextField[name = NIF,width=6cm]{} 
        \\ 
        \hline
        Fecha: &    \TextField[name = Fecha,width=6cm]{} 
        \\ 
        \hline
        Firma: &     
        \\ &
        \\ &
        \\ &
        \\ \hline
    \end{tabular}
\end{table}
%%%%%%%%%%%%%%%%%%%%%%%%%%%%%%%%%%%%%%%%%%%%%%%
\newpage
\begin{abstract}
    \resumen
\end{abstract}
\tableofcontents
\newpage
% \newpage
% % %%%%%%%%%%%%%%%%%%%%%%%%%%%%%%%%%%%%%%%%%%%%%%%
% % %%%%%%%%%%%%%%%%%%%%%%%%%%%%%%%%%%%%%%%%%%%%%%%
% % %%%%%%%%%%%%%%%%%%%%%%%%%%%%%%%%%%%%%%%%%%%%%%%
% % %%%%%%%%%%%%%%%%%%%%%%%%%%%%%%%%%%%%%%%%%%%%%%%
% % %%%%%%%%%%%%%%%%%%%%%%%%%%%%%%%%%%%%%%%%%%%%%%%































\section{Conclusiones}
La implementación de pantallas térmicas en los invernaderos es una inversión rentable a largo plazo, con beneficios económicos, productivos y ambientales. Este estudio demuestra que es una solución viable para agricultores interesados en optimizar sus recursos.

\section{Referencias}
\begin{itemize}
    \item \href{https://www.idae.es/publicaciones/guia-de-energia-solar-termica-para-procesos-industriales}
    {Guía IDAE 033: Guía de Energía Solar Térmica para Procesos Industriales (edición v1.0)}.

    \item Artículo científico, Revista, Año.
\end{itemize}

\end{Form}
\end{document}
